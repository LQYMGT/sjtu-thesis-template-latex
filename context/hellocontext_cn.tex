\usemodule[zhfonts]
\zhfonts[rm, 11pt]

\setuphead[title][style=\bfc]
\setuphead[section, subject][style=\bfa]

\setupcombinedlist[content][alternative=c]

\starttext

\title{并行程序示例}

\placecontent

\blank
本文档介绍如何在π超级计算机上编译和提交并行作业任务。
π支持OpenMP、MPI、CUDA等并行编程模型。
再继续阅读本文档之前,您应该知道如何登录π、使用LSF提交作业、Module的基本概念。
下面几个文档可以帮助您完成准备工作:

\startitemize
\item 使用SSH登录高性能计算节点 \from[sshlogin] 
\item LSF作业管理系统使用方法 \from[lsf]
\item 使用Environment Module设置运行环境 \from[envmodule]
\stopitemize

\section{OpenMP示例}

Pi集群上GCC和Intel编译器都支持OpenMP扩展。
示例代码omp_hello.c内容如下:

	\starttyping
	#include <omp.h>
	#include <stdio.h>
	#include <stdlib.h>
	
	int main (int argc, char *argv[])
	{
	int nthreads, tid;
	
	/* Fork a team of threads giving them their own copies of variables */
	#pragma omp parallel private(nthreads, tid)
	  {
	
	  /* Obtain thread number */
	  tid = omp_get_thread_num();
	  printf("Hello World from thread = %d\n", tid);
	
	  /* Only master thread does this */
	  if (tid == 0)
	    {
	    nthreads = omp_get_num_threads();
	    printf("Number of threads = %d\n", nthreads);
	    }
	
	  }  /* All threads join master thread and disband */
	
	}
	\stoptyping

\subsection{使用GCC编译器}

GCC编译OpenMP代码时加上```-fopenmp```:

	\starttyping
	$ gcc -fopenmp omp_hello.c -o ompgcc
	\stoptyping

在本地使用4线程运行程序:

	\starttyping
	$ export OMP_NUM_THREADS=4 && ./ompgcc
	\stoptyping

正式运行时需要提交到LSF作业管理系统,提交脚本```ompgcc.lsf```如下,仍使用4线程运行(增加约束条件让所有线程分配到一台物理机上):

	\starttyping
	#BSUB -L /bin/bash
	#BSUB -J HELLO_OpenMP
	#BSUB -n 4
	#BSUB -e %J.err
	#BSUB -o %J.out
	#BSUB -R "span[hosts=1]"
	#BSUB -q cpu
	
	export OMP_NUM_THREADS=4
	./ompgcc
	\stoptyping

\stoptext

\useURL[sshlogin][http://pi.sjtu.edu.cn/docs/SSH_ch]
\useURL[lsf][http://pi.sjtu.edu.cn/docs/LSF_ch]
\useURL[envmodule][http://pi.sjtu.edu.cn/docs/Module_ch]
